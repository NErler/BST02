% !TeX spellcheck = en_US
% !TeX program = lualatex
\documentclass[14pt, aspectratio=169, sectionpage=simple, xclolor=table]{beamer}
\usetheme{emctria}
\usepackage{colortbl}
\usepackage{fancyvrb}
\usepackage[autostyle]{csquotes}
\usetikzlibrary{calc,fadings,shapes.arrows,shadows,backgrounds, positioning}
\usepackage{adjustbox}
\usepackage{xcolor}
\usepackage{adjustbox}
\usepackage{verbatim}
\usepackage{listings}

\setlength{\unitlength}{1cm}
\newcommand{\ba}{\mbox{{\boldmath $a$}}}
\newcommand{\bA}{\mbox{{\boldmath $A$}}}
\newcommand{\bb}{\mbox{{\boldmath $b$}}}
\newcommand{\bB}{\mbox{{\boldmath $B$}}}
\newcommand{\bc}{\mbox{{\boldmath $c$}}}
\newcommand{\bC}{\mbox{{\boldmath $C$}}}
\newcommand{\bd}{\mbox{{\boldmath $d$}}}
\newcommand{\bD}{\mbox{{\boldmath $D$}}}
\newcommand{\be}{\mbox{{\boldmath $e$}}}
\newcommand{\bE}{\mbox{{\boldmath $E$}}}
\newcommand{\bbf}{\mbox{{\boldmath $f$}}}
\newcommand{\bF}{\mbox{{\boldmath $F$}}}
\newcommand{\bg}{\mbox{{\boldmath $g$}}}
\newcommand{\bG}{\mbox{{\boldmath $G$}}}
\newcommand{\bh}{\mbox{{\boldmath $h$}}}
\newcommand{\bH}{\mbox{{\boldmath $H$}}}
\newcommand{\bi}{\mbox{{\boldmath $i$}}}
\newcommand{\bI}{\mbox{{\boldmath $I$}}}
\newcommand{\bfI}{\mathbf{I}}
\newcommand{\bj}{\mbox{{\boldmath $j$}}}
\newcommand{\bJ}{\mbox{{\boldmath $J$}}}
\newcommand{\bk}{\mbox{{\boldmath $k$}}}
\newcommand{\bK}{\mbox{{\boldmath $K$}}}
\newcommand{\bl}{\mbox{{\boldmath $l$}}}
\newcommand{\bL}{\mbox{{\boldmath $L$}}}
\newcommand{\bm}{\mbox{{\boldmath $m$}}}
\newcommand{\bM}{\mbox{{\boldmath $M$}}}
\newcommand{\bn}{\mbox{{\boldmath $n$}}}
\newcommand{\bN}{\mbox{{\boldmath $N$}}}
\newcommand{\bo}{\mbox{{\boldmath $o$}}}
\newcommand{\bO}{\mbox{{\boldmath $O$}}}
\newcommand{\bp}{\mbox{{\boldmath $p$}}}
\newcommand{\bP}{\mbox{{\boldmath $P$}}}
\newcommand{\bq}{\mbox{{\boldmath $q$}}}
\newcommand{\bQ}{\mbox{{\boldmath $Q$}}}
\newcommand{\br}{\mbox{{\boldmath $r$}}}
\newcommand{\bR}{\mbox{{\boldmath $R$}}}
\newcommand{\bs}{\mbox{{\boldmath $s$}}}
\newcommand{\bS}{\mbox{{\boldmath $S$}}}
\newcommand{\bt}{\mbox{{\boldmath $t$}}}
\newcommand{\bT}{\mbox{{\boldmath $T$}}}
\newcommand{\bu}{\mbox{{\boldmath $u$}}}
\newcommand{\bU}{\mbox{{\boldmath $U$}}}
\newcommand{\bv}{\mbox{{\boldmath $v$}}}
\newcommand{\bV}{\mbox{{\boldmath $V$}}}
\newcommand{\bw}{\mbox{{\boldmath $w$}}}
\newcommand{\bW}{\mbox{{\boldmath $W$}}}
\newcommand{\bx}{\mbox{{\boldmath $x$}}}
\newcommand{\bX}{\mbox{{\boldmath $X$}}}
\newcommand{\by}{\mbox{{\boldmath $y$}}}
\newcommand{\bY}{\mbox{{\boldmath $Y$}}}
\newcommand{\bz}{\mbox{{\boldmath $z$}}}
\newcommand{\bZ}{\mbox{{\boldmath $Z$}}}

\newcommand{\bfalpha}{\mbox{{\boldmath $\alpha$}}}
\newcommand{\bfbeta}{\mbox{{\boldmath $\beta$}}}
\newcommand{\bfgamma}{\mbox{{\boldmath $\gamma$}}}
\newcommand{\bfdelta}{\mbox{{\boldmath $\delta$}}}
\newcommand{\eps}{\varepsilon}
\newcommand{\bfeps}{\mbox{{\boldmath $\varepsilon$}}}
\newcommand{\bfeta}{\mbox{{\boldmath $\eta$}}}
\newcommand{\bftheta}{\mbox{{\boldmath $\theta$}}}
\newcommand{\bfmu}{\mbox{{\boldmath $\mu$}}}
\newcommand{\bfpi}{\mbox{{\boldmath $\pi$}}}
\newcommand{\bfsigma}{\mbox{{\boldmath $\sigma$}}}
\newcommand{\bfomega}{\mbox{{\boldmath $\omega$}}}
\newcommand{\bfPi}{\mbox{{\boldmath $\Pi$}}}
\newcommand{\bfSigma}{\mbox{{\boldmath $\Sigma$}}}
\newcommand{\bfPhi}{\mbox{{\boldmath $\Phi$}}}


\newcommand{\logit}{\mbox{logit}}
\newcommand{\expit}{\mbox{expit}}

\newcommand{\code}[1]{\textcolor{blue}{\texttt{#1}}}
\newcommand{\red}[1]{\textcolor{red}{#1}}
\newcommand{\blue}[1]{\textcolor{blue}{#1}}
\newcommand{\redbf}[1]{\textcolor{red}{\textbf{#1}}}
\newcommand{\bluebf}[1]{\textcolor{blue}{\textbf{#1}}}
\newcommand{\magentabf}[1]{\textcolor{magenta}{\textbf{#1}}}
\newcommand{\argmax}{\operatornamewithlimits{argmax}}
\newcommand{\rightarrowP}{\operatornamewithlimits{\rightarrow}}
\newcommand{\etal}{{\it et al., }}
\newcommand{\Rnsp}{\textsf{R}}
\newcommand{\R}{{\textsf{R} }}


\newcommand{\nl}{{\vspace{1ex} }}

\tikzset{bubble/.style={rectangle, draw=gray,rounded corners,fill=#1,align = flush center,minimum height=1cm,minimum width=1.75cm}}

\tikzfading [name=arrowfading, top color=transparent!0, bottom color=transparent!95]
\tikzset{arrowfill/.style={top color=red!20, bottom color=red, general shadow={fill=black, shadow yshift=-0.8ex, path fading=arrowfading}}}
\tikzset{arrowstyle/.style={draw=red!10,arrowfill, single arrow,minimum height=#1, single arrow,
    single arrow head extend=.4cm,}}

\newenvironment{code11}%
{\par\noindent\adjustbox{margin=1ex,bgcolor=emclblue!20,%
		margin=0ex \medskipamount}\bgroup\varwidth\linewidth\verbatim}%
{\endverbatim\endvarwidth\egroup}

\date{25 February - 1 March 2019}

%\let \nl = \vspace{1ex}

%\renewcommand{\arraystretch}{2}

%\addtolength{\oddsidemargin}{1cm}



\title{Using R for Statistics in \\ Medical Research}
\author{{\large \textbf{Elrozy Andrinopoulou}}\\
	\small{\texttt{e.andrinopoulou@erasmusmc.nl}} \vspace*{1ex} \\
	\and 
	{\large 	\textbf{Sten Willemsen}}\\
	\small{\texttt{s.willemsen@erasmusmc.nl}}}
\institute{Department of Biostatistics, Erasmus University Medical Center}

\newenvironment{code12}%
{\par\noindent\adjustbox{margin=1ex,bgcolor=emclblue,margin=0ex \medskipamount}\bgroup\verbatim}%
{\endverbatim\egroup}

\begin{document}
	
	%\subtitle{}
	%\logo{}
	%\institute{}
	%\date{}
	%\subject{}
	%\setbeamercovered{transparent}
	%\setbeamertemplate{navigation symbols}{}
	
\begin{titleframe}
\end{titleframe}
%======================================================
%======================================================
\section{Practical 1}
%======================================================
\begin{frame}{Practical 1.1}
\begin{itemize}
\item Create 2 vectors of size 50
\begin{itemize}
\item Sex: takes 2 values 0 and 1 (hint: use \code{sample()})
\item Age: takes values from 20 till 80 (hint: use \code{sample()} to sample from 20:80)
\item convert \enquote{Sex} to a factor with levels 0: \enquote{female} and 1: \enquote{male}
\item define the new variable \enquote{AgeCat} as dichotomous with \enquote{Age}$<=$50 to be 0 and 1 otherwise
\item convert \enquote{AgeCat} to a factor with levels 0: \enquote{young} and 1: \enquote{old}
\item overwrite \enquote{Age} variable by $\frac{Age-mean(Age)}{sd(Age)}$
\end{itemize}
\end{itemize}
\mbox{\redbf{Save your code}}
\end{frame}

\begin{frame}{Practical 1.2}
\begin{itemize}
\item Create a data.frame with the name \enquote{DF}
\begin{itemize}
\item including the following vectors: \enquote{Sex,} \enquote{Age}, \enquote{AgeCat}
\item with names: \enquote{Gender}, \enquote{StandardizedAge}, \enquote{DichotomousAge}
\item What are the dimensions of the data.frame (hint: use \code{dim()})?
\end{itemize}
\end{itemize}
\end{frame}

\begin{frame}{Practical 1.2.1 \redbf{-extra}}
\begin{itemize}
\item Create 2 vectors of size 150
\begin{itemize}
\item Treatment: takes 2 values 1 and 2
\item Weight: takes values from 50 till 100
\item convert \enquote{Treatment} to a factor with levels 1: \enquote{no} and 2: \enquote{yes}
\item overwrite \enquote{Weight} variable by \enquote{Weight}$*1000$
\item create a data.frame including \enquote{Treatment} and \enquote{Weight}
\end{itemize}
\end{itemize}
\end{frame}

\begin{frame}{Practical 1.2.2 \redbf{-extra}}
\begin{itemize}
\item Create a list called \enquote{my\_list} with the following:
\begin{itemize}
\item let: \enquote{$a$} to \enquote{$i$} (hint: use \code{letters})
\item mat: matrix $\binom{1\hspace{0.4cm}2}{3\hspace{0.3cm}4}$
\item sex: factor taking the values \enquote{males} and \enquote{females} and length 50
\end{itemize}
\end{itemize}
\end{frame}
%======================================================
\section{Practical 2}
%======================================================
\begin{frame}{Practical 2.1}
\begin{itemize}
\item Using the data.frame \enquote{DF} from Practical 1
\begin{itemize}
\item calculate the mean of the variable \enquote{StandardizedAge}
\item calculate the standard deviation of the variable \enquote{StandardizedAge}
\item calculate the frequencies of the variable \enquote{Gender} (hint: use function  \code{length()} or \code{table()})
\item calculate the frequencies of the variable \enquote{DichotomousAge}
\end{itemize}
\end{itemize}
\end{frame}
%======================================================
\begin{frame}{Practical 2.2}
\begin{itemize}
\item Using the data.frame \enquote{DF} from Practical 1
\begin{itemize}
\item calculate the mean of the variable \enquote{StandardizedAge} for males and females
\item show the crosstab including \enquote{Gender} and \enquote{DichotomousAge} (hint: use function \code{table()})
\item select all males that are young
\item select all females or old patients
\end{itemize}
\end{itemize}
\end{frame}
%======================================================
\begin{frame}{Practical 2.2.1 \redbf{-extra}}
\begin{itemize}
\item Using the data.frame \enquote{DF} from Practical 1
\begin{itemize}
\item calculate the median of the variable \enquote{StandardizedAge}
\item calculate the median of the variable \enquote{StandardizedAge} for males and females
\item show the 2nd column of DF
\end{itemize}
\end{itemize}
\end{frame}
%======================================================
\section{Practical 3}
%=====================================
\begin{frame}{Practical 3.1}
\begin{itemize}
\item Create a function with the name \enquote{Descriptives} that
\begin{itemize}
\item takes as values 2 vectors
\item calculates the min, median, max, mean, sd and length of both vectors
\item returns a matrix where rows indicate the statistic and columns the vectors (hind: to name the rows of the matrix use \code{rownames()})
\end{itemize}
\item Use the function for the following vectors
\begin{itemize}
\item vec1: sample \enquote{age} from 20 till 80 and with n = 100
\item vec2: sample \enquote{weight} from 60 till 100 and with n = 200
\end{itemize}
\end{itemize}
\end{frame}
%======================================================
\begin{frame}{Practical 3.1.1 \redbf{-extra}}
\begin{itemize}
\item Create a function with the name \enquote{Summaries} that
\begin{itemize}
\item takes as value a data.frame that includes 2 columns: 1st categorical, 2nd continuous
\item calculates the mean and median per group (variable in 1st column) for the variable in the 2nd column
\item returns a matrix where rows indicate the mean/median and columns the groups
\end{itemize}
\item Use the function for the data.frame that includes the following
\begin{itemize}
\item vec1: sample \enquote{age} from 20 till 80 and with n = 100
\item vec2: sample \enquote{sex} takes 0 and 1, with n = 100 (convert also to a factor)
\end{itemize}
\end{itemize}
\end{frame}
%======================================================
\section{Practical 4}
%======================================================
\begin{frame}{Practical 4.1}
\begin{itemize}
\item Import the spss dataset: Cancer.sav
\begin{itemize}
\item we are interested in testing if there is a difference in the average age between the 2 treatment groups (t-test)
\item we obtained the result, but it was nowhere saved! Save the result of the t-test in an object named 'res' for post processing
\item using indexing extract the p-value
\item using indexing extract the 95\% CI
\end{itemize}
\end{itemize}
\end{frame}
%======================================================
\begin{frame}{Practical 4.2}
\begin{itemize}
\item We want to save those results for reporting in our paper. Create a data.frame with the name \enquote{Restest}, with columns
\begin{itemize}
\item mean1: mean for TRT = 0
\item mean2: mean for TRT = 1
\item diffres: difference of the means
\item lowCI: lower limit of confidence interval
\item upCI: upper limit of confidence interval
\item pVal: p-value of the test
\end{itemize}
\end{itemize}
\end{frame}
%======================================================
\begin{frame}{Practical 4.3}
\begin{itemize}
\item Using the dataset: Cancer.sav
\begin{itemize}
\item create a dichotomous variable for \enquote{WEIGHIN}: 0 if \enquote{WEIGHIN} $<$ mean(\enquote{WEIGHIN}) and 1 otherwise. Use the name \enquote{WEIGHINcat}
\item perform a t-test comparing \enquote{AGE} between the \enquote{TRT} groups
\item perform a t-test comparing \enquote{AGE} between the \enquote{WEIGHINcat} groups
\item create a vector with the name \enquote{results}, including the p-values of the above tests
\item calculate the significance level after the bonferroni correction. Use the name \enquote{bonf.corr\_alpha}
\item using an if statement, print whether the null hypothesis is rejected or not
\end{itemize}
\end{itemize}
\end{frame}
%======================================================
\begin{frame}{Practical 4.4}
\begin{itemize}
\item Using the Cancer.sav dataset
\begin{itemize}
\item perform a regression analysis with dependent variable: WEIGHIN and independent: TRT - use the name fm1
\item perform a regression analysis with dependent variable: WEIGHIN and independent: TRT + STAGE - use the name fm2
\item compare the 2 models
\item perform a regression analysis with dependent variable: WEIGHIN and independent: TRT + STAGE + AGE - use the name fm3
\item include an interaction term of AGE and STAGE in the previous model - use the name fm4
\end{itemize}
\end{itemize}
\end{frame}
%======================================================
\begin{frame}{Practical 4.5}
\begin{itemize}
\item Using the Cancer.sav dataset
\begin{itemize}
\item create a function with the name \enquote{Pred} that takes as values: the treatment, stage and age of a new patient and returns the predicted weight using model fm4
\begin{itemize}
\item[-] Predict the weight of a new patient: treatment = 1, stage = 2 and age = 77
\end{itemize}
\item plot the weight vs the age per treatment group (xyplot with 2 figures). The x-axis will have the name \enquote{age}, the y-axis will have the name \enquote{weight} and the main title will be \enquote{weight vs age}
\end{itemize}
\end{itemize}
\end{frame}
%======================================================
\begin{frame}{Practical 4.5.1 \redbf{-extra}}
\begin{itemize}
\item Create the following vectors of size 100
\begin{itemize}
\item sex: takes 2 values 1 and 2
\item tr: takes 4 values 1 to 4
\item score: normal distribution with mean 20 and sd 5
\end{itemize}
\item Create boxplots of score per sex and tr groups
\item Replace the above vectors using n = 20 instead of 100
\item Create new boxplots of score per sex and tr groups
\item What can you say about the distrubution?
\end{itemize}
\end{frame}
%======================================================
\begin{frame}{Practical 4.5.2 \redbf{-extra}}
\begin{itemize}
\item Using the Cancer.sav dataset
\begin{itemize}
\item what is the Pearson correlation between \enquote{WEIGHIN} and \enquote{AGE}?
\item perform the test, what is hypothesis and the decision?
\item create a scatterplot to support your findings
\item create a boxplot and investigate the col argument
\end{itemize}
\end{itemize}
\end{frame}
%======================================================
\section{Practical 5}
%======================================================
\begin{frame}{Practical 5.1}
\begin{itemize}
\item Using the Cancer.sav dataset, create a function \enquote{uniRegr} that
\begin{itemize}
\item takes a data.frame and a vector with the names of the independent covariates as c(\enquote{ }, \enquote{ }, \enquote{ }, \dots)
\item performs univariable regression analysis of the outcome:\enquote{WEIGHIN} and each covariate (hint: use \code{as.formula()} and \code{paste()})
\item returns the estimate, std error and the p-values as a data.frame
\end{itemize}
\item Using the Cancer.sav dataset, perform univariable regression analysis using the covariates \enquote{AGE}, \enquote{STAGE} and \enquote{TRT}
\item Save the results as \enquote{results\_regr}
\end{itemize}
\end{frame}
%======================================================
\begin{frame}{Practical 5.1.1 \redbf{-extra}}
\begin{itemize}
\item Using the Cancer.sav dataset, create a function \enquote{uniLogRegr} that
\begin{itemize}
\item takes a data.frame and a vector with the names of the independent covariates as c(\enquote{ }, \enquote{ }, \enquote{ }, \dots)
\item performs univariable \bluebf{logistic} regression analysis of the outcome:\enquote{TRT} and each covariate (hint: use \code{as.formula()} and \code{paste()})
\item returns the estimate, std error and the p-values as a data.frame
\end{itemize}
\item Using the Cancer.sav dataset, perform univariable logistic regression analysis using the covariates \enquote{AGE}, \enquote{STAGE} and \enquote{WEIGHIN}
\item Save the results as \enquote{results\_log\_regr}
\end{itemize}
\end{frame}


\label{einde document}
\end{document}
